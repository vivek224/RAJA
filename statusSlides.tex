\documentclass{beamer}
\usetheme{Dresden}


\usepackage{listings}

\usepackage{graphicx}
\usepackage{color}
\usepackage{url}

\definecolor{codegreen}{rgb}{0,0.6,0}
\definecolor{codegray}{rgb}{0.5,0.5,0.5}
\definecolor{codepurple}{rgb}{0.58,0,0.82}
\definecolor{backcolour}{rgb}{0.95,0.95,0.92}




 \lstset{
        language=C++,
        basicstyle=\tiny\ttfamily\color{blue}, % Standardschrift %                                                                                                                                                                 
                                    % % TODO: consider making tiny to make                                                                                                                                                         
                                     % % things fit                                                                                                                                                                                
        numbers=left, % Ort der Zeilennummerg                                                                                                                                                                                      
         numberstyle=\footnotesize\ttfamily,          % Stil der Zeilennummern                                                                                                                                                     
         stepnumber=2,               % Abstand zwischen den Zeilennummern                                                                                                                                                          
         numbersep=6pt,              % Abstand der Nummern zum Text                                                                                                                                                                
         tabsize=2,                  % Groesse von Tabs                                                                                                                                                                            
         extendedchars=true,         %                                                                                                                                                                                             
         breaklines=true,            % Zeilen werden Umgebrochen                                                                                                                                                                   
         frame=single,
         keywordstyle=[1]\textbf,% Stil der Keywords                                                                                                                                                                              
         keywordstyle=\color{red},
         stringstyle=\ttfamily,
         showspaces=false,           % Leerzeichen anzeigen ?                                                                                                                                                                     
         showtabs=false,             % Tabs anzeigen?                                                                                                                                                                              
         showstringspaces=false,    % Leerzeichen in Strings anzeigen ?                                                                                                                                                            
         commentstyle=\color{codegreen},
         numberstyle=\tiny\color{codegray},
         stringstyle=\color{codepurple},
         commentstyle=\color{green},
         morecomment=[l][\color{magenta}]{\#},
         captionpos=b
}

\lstdefinestyle{mystyle}{
     backgroundcolor=\color{backcolour},
    commentstyle=\color{codegreen},
    keywordstyle=\color{magenta},
    numberstyle=\tiny\color{codegray},
    stringstyle=\color{codepurple},
    basicstyle=\footnotesize,
    breakatwhitespace=false,
    breaklines=true,
    captionpos=b,
    keepspaces=true,
    numbers=left,
    numbersep=5pt,
    showspaces=false,
    showstringspaces=false,
    showtabs=false,
    tabsize=2
 }



\setbeamertemplate{footline}[frame number]

\newcommand{\comments{}}[1]{}

\title{Status of RAJA fork Having Low-overhead Loop Scheduling Integrated within RAJA, i.e., lws-RAJA}
\author{Vivek Kale}
\date{\today} 

\begin{document} 

\begin{frame}
\titlepage
\end{frame}

\begin{frame}[label=ovwlwsRAJAstatus]{Overview of Slides for Status of lws-RAJA}
\begin{enumerate}
\item A need: Add new scheduling strategies to a library with abstractions for loops for scientific and engineering applications run on a supercomputer.
\item What is trying to be done: integrate loop scheduling strategy library in RAJA.
\item How it's being done: specific scheme for Implementation 
\item Failed Test cases with errors (the result of doing it) 
\item Action items include fixing errors and testing more (Conclusions and what needs to be done next)
\item Todo and potential schedule: add more strategies, test with other app codes
\end{enumerate}
\end{frame}

\begin{frame}{Need Sophisticated Scheduling Strategies in RAJA}
\begin{itemize}
\item Need a larger number of loop scheduling strategies in RAJA that application programmers can experiment with.
\item Also, Vivek is interested in doing to learn more about and get experience with developing a tool like RAJA. 
\end{itemize} 
\end{frame}


\begin{frame}{Example Code with RAJA}
\begin{figure}[ht!] 
\lstinputlisting{listings/appFor_RAJASimple_loop.c}
\end{figure}
\end{frame}

\begin{frame}{Implementation of Abstractions with RAJA}
\begin{figure}[ht!] 
\lstinputlisting{listings/appFor_RAJASimple_forall.c}
\end{figure}
\end{frame}

\begin{frame}{Example Code with lw-sched}
\begin{figure}[ht!] 
\lstinputlisting{listings/appFor_vSchedSimple_loop.c}
\end{figure}
\end{frame}

\begin{frame}{Implementation of Abstractions with lw-sched}
\begin{figure}[ht!] 
\lstinputlisting{listings/appFor_vSchedSimple_forall.c}
\end{figure}
\end{frame}

\begin{frame}{Integrate Loop Scheduling Strategy Library in RAJA}
\begin{itemize}
\item Use abstractions of RAJA to reduce lines of code for an application programmer to write a program with loops 
\item Bring up to the abstraction layer the ability for the application programmer to change the loop scheduling scheme, allowing the application programmer to only have to think about efficiency of loop scheduling in terms of loop iterations and threads/cores.  
\item The result is better performing application code in RAJA because more scheduling strategies are allowed. 
\end{itemize} 
\end{frame}

\begin{frame}{Example Code with RAJA fork of lw-sched}
\begin{figure}[ht!] 
\lstinputlisting{listings/appFor_lws-RAJASimple_loop.c}
\end{figure}
\end{frame}

\begin{frame}{Implementation of Abstractions with RAJA fork of lw-sched}
\begin{figure}[ht!] 
\lstinputlisting{listings/appFor_lws-RAJASimple_forall.c}
\end{figure}
\end{frame}

\begin{frame}{Scheme for Implementing Integration}
\begin{itemize}
\item Change the code {\tt forall.hpp} and add the function {\tt loop\_next()} within it as suggested by David Beckingsale during ATPESC 2018 and started then.
\item Link my library with RAJA by modifying RAJA's {\tt CmakeLists.txt.}
\item Add {\tt vSched.c} and {\tt vSched.h}, along with a Makefile and the test cases I've done for my library previously(should be unit tests in the context of this repo), in repository for my fork of RAJA for convenience.
\end{itemize}
\end{frame}

\begin{frame}[allowframebreaks]{Results of Initial Integration: Test Cases with Errors}
\begin{itemize}
\small \item \small My library doesn't Maximum size of array of floats only. Need to handle other data types, which RAJA (I think does). $\rightarrow$ Need to define the responsibility of my library and RAJA library.
\item \small How do we handle cases number of loop iterations of a loop (or number of chunks of a loop) is less than than the number of threads?
\item \small In regards to previous test case, how do we handle the case when dynamic fraction is greater than 50\% and the number of threads is greater than the number of scheduled iterations (all iterations dynamically scheduled). In the case when dynamic fraction is less 50\% dynamic fraction, we can do an assignment of iterations to threads statically in a predefined manner to avoid division by zero (problem has been solved in another variant library I'm working on).
\item \small In regards to previous test case, how do we handle the case when dynamic fraction is greater than 50\% and the number of threads is greater than the number of scheduled iterations (all iterations dynamically scheduled). The case when we have less than 50\% dynamic fraction?
\item \small We want to make sure critical and fundamental errors are handled completely before thinking about others like nested loops $\rightarrow$ deal with nested loops later. 
\item \small Have consider kinds of optimizations one does for GPU for lw-sched, can I leave these? Is CHAI the appropriate library?
\end{itemize}
\end{frame}

\begin{frame}{Action Items}
\begin{itemize}
\item Need to fix errors specific to my library with respect to cases that RAJA handles. 
\item Need to define the responsibility of my library and the responsibilities of RAJA library.
\item Based on these responsibilities, need to handle test cases for lw-sched micro-library, which is put in the RAJA library. 
\end{itemize}
\end{frame}

\begin{frame}{To Do and Schedule}
\begin{itemize}
\item Add more scheduling strategies once basic scheduling strategy is tested
\item Test with Sparse Matrix Vector Multiplication (having load imbalance across loop iterations).
\item Do this in the next three weeks, maybe can use this in my booth talk at SC 2018 on overviews and talks about user-defined schedules for loop scheduling in OpenMP.
\end{itemize}
\end{frame}


\end{document}
